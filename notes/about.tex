\documentclass{article}
\usepackage{amsmath, amsfonts, amssymb}
\usepackage{graphicx}
\usepackage{titlesec}
\usepackage{lipsum}
\usepackage[utf8]{inputenc}
\usepackage{xcolor}
\usepackage{listings}
\usepackage{lipsum}
\usepackage{scrextend}
\usepackage{tikz}
\usepackage{hyperref}
\usepackage{color}
\usepackage[left=2.5cm,top=3cm,right=2.5cm,bottom=3cm,bindingoffset=0.5cm]{geometry}

\title{About}
\author{Luc Alexander L{\"u}thi}
\date{December 26th, 2025}

\begin{document}
	\maketitle
	\tableofcontents
	\section{Who am I?}

	I'm a programmer and independent researcher. 

	\section{What am I Doing?}
	
	I am researching metacompilation, guided program synthesis, and adversarial programming. 

	\section{Metacompilation}
	Meta computation is reasoning about programs, when we compile metaprograms we are compiling programs into other programs rather into an output ready format. This allows us to reason about computation on a higher level and gives us more control over both the architecture of our program and its literal machine readable makeup. A large focus of my research is how to use meta computing tools to express and test systems of constraint. 

	\section{Guided Program Synthesis}
	Guided program synthesis is a way to generate program text which fulfils a set of invariants. This allows us to generate arbitrary programs which fulfil tasks under a given system of constraint.

	\section{Adversarial Programming}
	Adversarial programming involves writing programs with the express purpose of breaking the invariants of the system which constrains the programmer. Whether this system is a type system, a logical proof checker, a borrow checker, a restriction of the program state to a finite state automota, or something entirely dynamic. 

\end{document}
